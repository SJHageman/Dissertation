\chapter{Conclusion}
\label{chap:conclusion}


The work presented herein represents the culmination of many years worth of research, and it sets the foundation for future experiments. In particular, demonstrating the ability to directly measure the full complex refractive index of ground states and excited states opens the door to a new class of experiments that were not previously possible.  Each Chapter builds towards this crescendo, starting with Chapter \ref{chap:beamline} where the TABLe was introduced.  This beamline was designed to have the unique capabilities necessary to perform the experiments that would follow.  Starting with the experiments done in Chapter \ref{chap:two_source}, which demonstrated the use of a 0-$\pi$ SWPG to generate two XUV sources and control the relative phase between them.  This interferometric control is critical to the retrieval of the complex refractive index that was done in Chapter \ref{chap:refractive_index}.  In this Chapter, it was demonstrated that the SWPG can be used to measure the ground state refractive index over a broad energy range of two different materials.  This measurement of the ground state refractive index naturally begs the question: Can the excited state be measured too?  To study this, a suitable physical system was selected for study, and that was the argon Fano resonances.  An initial experiment was performed using standard transient absorption to fully characterize this system in our experimental conditions, and this was detailed in Chapter \ref{chap:ATS}.  Following this, the methods developed in Chapter \ref{chap:refractive_index} were extended to two-source transient absorption, and this was termed the CATS method.  CATS was used to measure the excited state dynamics of the laser dressed argon Fano resonances in Chapter \ref{chap:CATS}.  The success of CATS in this study is what conclusively demonstrates the capability of CATS to be a viable method for future experiments to gain access to both the real and imaginary parts of the refractive index.  CATS was demonstrated in a gas phase experiment, however it can easily be applied to condensed matter experiments in either a transmission or a reflection geometry.  This allows for full and direct characterization of the induced dynamics, and this can be a powerful tool to further advance attosecond physics.





