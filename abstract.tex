\begin{abstract}

An experimental technique is developed to measure the complex refractive index in a transient-absorption experiment using attosecond pulse trains.  This complex attosecond transient-absorption spectroscopy (CATS) method is demonstrated by measuring the dynamic change, induced by an infrared dressing pulse, in the real and imaginary parts of the refractive index of the argon $3s3p^6np$ Fano resonances.  Typical attosecond transient-absorption spectroscopy (ATS) measurements only capture the imaginary part of the refractive index, and the real part can only be indirectly calculated.  CATS enables a direct measurement of the real part of the refractive index, and this removes the need to rely upon indirect calculations which are only valid if certain assumptions hold true.  While CATS is demonstrated in a gas phase experiment, it can also be used for condensed matter experiments in either a transmission or reflection geometry.

As a prelude to the demonstration of CATS, an ATS experiment is performed to examine the dynamics of the argon $3s3p^6np$ Fano resonances under the influence of a dressing field.  This ATS measurement reveals a complicated structure of light-induced states and light-induced attenuation in the intensity and time delay dependence of the absorption spectrum.  The theoretical understanding of these features is detailed, and excellent agreement between theoretical and experimental results is demonstrated. 

Additionally, the optical tool that enables CATS to be performed is detailed theoretically and experimentally. This tool is a diffractive optical element known as a $0-\pi$ square-wave phase grating (SWPG).  The SWPG allows for an input femtosecond IR pulse to be duplicated, and the SWPG can control the relative phase between these duplicates.  This relative phase control is demonstrated, and it is used to measure the ground state complex refractive index of silicon and germanium.


\end{abstract}
